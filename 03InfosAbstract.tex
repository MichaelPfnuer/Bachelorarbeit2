\chapter*{Allgemeine Informationen}
\thispagestyle{plain}
\pagestyle{plain}

\begin{tabular}{p{0.3\textwidth}p{0.65\textwidth}}

Vor- und Zuname: & \Author \\*[0.2cm]
Institution: & Fachhochschule Salzburg GmbH \\*[0.2cm]
Studiengang: & Informationstechnik \& System-Management \\*[0.2cm]
Titel der Bachelorarbeit: & \Title \\*[0.2cm]
Schlagwörter: & \Keywords  \\*[0.2cm]
Betreuer an der FH: & \Advisor

\end{tabular}
\newpage

\section*{\Large\bfseries Kurzzusammenfassung}

Single Board Computer (SBC) sind Einplatinenrechner, bei denen die verschiedenen Hardware Komponenten auf einer Platine zusammengefasst sind.\\
Diese Arbeit beschäftigt sich mit dem Aufbau verschiedener Schaltungen zum Einsatz in einem bestehenden Fertigungsumfeld. Zunächst wird auf die unterschiedlichen Möglichkeiten zur Datenübertragung von SBCs und Sensoren eingegangen und einige verschiedenen Übertragungssysteme näher erörtert.\\
Im praktischen Teil der Arbeit werden drei Testszenarien für den Einsatze von SBCs beschrieben und auf die unterschiedlichen Möglichkeiten zur Datenspeicherung und Visualisierung eingegangen. Die Unterschiede, sowie die Vor- und Nachteile der einzelnen Varianten zur Datenspeicherung werden hier dargelegt.
\vspace{1cm}\\
\section*{\Large\bfseries Abstract}

Single Board Computers are computers where the different hardware is placed on one platin.\\
This paper concerned with the construction of several circuits in relation to an  application in existing production lines. First of all the capabilities of communication from SBCs and sensors  as well as different  telecommunication systems will be explained.\\
The practical part of this paper showes three test settings for the application of SBCs and discribes the various possibilities respectively to data storage and visualization. The variaties, advantages and disadvantages of the different versions will be presented.