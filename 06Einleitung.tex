\chapter{Einleitung}

Ein häufig verwendetes Schlagwort heutzutage wenn über die industrielle Produktion gesprochen wird ist \textbf{Industrie 4.0}. Damit ist vornehmlich gemeint, dass viele Abläufe vollautomatisiert von statten gehen. Egal ob dies in der Produktion selbst, in der Logistik, bei der Materialbestellung oder auch beim Versand der Fall ist. Um diese Anforderungen zu bewerkstelligen werden immer neuere Technologien eingesetzt, in diesen Zusammenhang werden Produkte wie Kleinstrechner, sogenannte \textbf{\ac{SBC}} von immer größeren Interesse. Beispiele hierfür sind der \textbf{\glqq Raspberry Pi\grqq} oder auch der \textbf{\glqq Banana Pi\grqq}, um nur einmal zwei der bekanntesten zu nennen. Es gibt allerdings auch noch eine Vielzahl anderer Produkte von \acp{SBC} auf dem Markt.

Die Aufgabe dieser Arbeit bestand darin, die Möglichkeiten für einen Einsatz von solchen \acp{SBC} in einem bestehenden Produktionsumfeld zu erproben. Der Bereich für den Einsatz erstreckt sich von der Temperaturmessung in den einzelnen Maschinen einer Produktionslinie bis hin zur Temperaturmessung in Schaltschränke oder Serverräumen um etwaige zu hohe Temperaturen frühzeitig erkennen zu können und diesen entgegenzuwirken. Weiterhin sollten auch noch Möglichkeiten für den Einsatz von Feuchtigkeitssensoren oder Vibrationssensoren erarbeitet werden, um z.B. Aussagen über die Schwingungsbelastung von nahegelegenen vielbefahrenen Zugstrecken und deren eventuelle Auswirkung auf die Produktion tätigen zu können. Ein weiterer zu erarbeitender Punkt war unterschiedliche Möglichkeiten zu testen, um die von den Sensoren gelieferten Daten effektiv zu speichern und aufzubereiten.

Die Arbeit ist folgendermaßen gegliedert. In Kapitel zwei werden einige der sich auf dem Markt befindenden \acp{SBC} miteinander verglichen um deren Vor- und Nachteile darzulegen und die bestmögliche Variante für die gegebenen Anforderungen auswählen zu können. Weiterhin werden die in den folgenden Kapiteln verwendeten Fachbegriffe erklärt, um diese zu verstehen. Auch werden die im späteren Verlauf verwendeten Bussysteme zur Übertragung der Daten vom Sensor an den \acl{SBC} erläutert, sowie dessen Übertragungsprotokolle. Im dritten Kapitel dem praktischen Teil werden die verschiedenen Schaltungen von unterschiedlichen Sensoren, sowie den unterschiedlichen Möglichkeiten zur Speicherung und Visualisierung der Daten dargestellt. Dabei wird vor allem auf die Datenspeicherung mittels \textbf{mySQL} und mit dem \textbf{RRDTool} eingegangen. Im letzten, den vierten Kapitel werden die zuvor erlangten Ergebnisse noch einmal zusammengefasst und ein Ausblick auf die Verwendung von \acp{SBC} in den verschiedenen Einsatzbereichen für die Zukunft gegeben.