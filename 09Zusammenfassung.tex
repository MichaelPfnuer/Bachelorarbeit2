\chapter{Zusammenfassung und Ausblick}
\label{chapter Zusammenfassung und Ausblick}
Diese Arbeit befasste sich damit, die Einsatzmöglichkeiten von SBCs in einem vorhandenen Fertigungsumfeld genauer zu betrachten. Dabei wurde sich speziell auf die Bereiche Temperatur- und Vibrationserfassung fokussiert und den damit verbundenen Möglichkeiten zur Speicherung und Visualisierung der Sensordaten. Eine der ersten Aufgabe war es, die bestmöglichen Komponenten bezüglich SBC und Sensoren auszuwählen um ein möglichst effektives und genaues Messergebnis zu erlangen. Die dafür notwendigen Informationen wurden durch eine Vielzahl von Recherchen erzielt, da es in diesem Bereich wenig bis gar keine Erfahrung gibt.\\
Nach der Auswahl der Komponenten an Hand bestimmter Faktoren wie Preis, Schnittstellen, Einsatzmöglichkeiten etc. sollte der Betrieb durch verschiedene Prototypen praktisch auf einem Steckbrett ausgetestet werden. Hierfür wurden drei verschiedene Schaltungen wie in Abschnitt \ref{section_DS18S20}, \ref{section_HTY221} und \ref{section_BMA020} beschrieben aufgebaut und betrieben.\\\\
Nachdem das Lesen der Daten von den verschiedenen Sensoren möglich war, wurde sich mit der Datenspeicherung befasst. Die zwei dabei vorgestellten Methoden wurden für die verschiedenen Zwecke getestet und ausgewertet. Am Ende kann der Schluss gezogen werden, dass sich das RRDtool sehr gut zur Erfassung von Temperatur- und Feuchtigkeitsdaten über einen längeren Zeitraum eignet, da es gleichzeitig die Möglichkeit bietet aus diesen Daten die entsprechenden Grafiken zu erstellen. Allerdings stößt es an seine Grenzen, wenn es darum geht Daten in nahezu Echtzeit wiederzugeben. 
\\\\Die zweite Möglichkeit der Datenspeicherung mittels einer mySQL Datenbank erweist sich dann als sinnvoll, wenn die Daten zu einem späteren Zeitpunkt weiterverarbeitet werden sollen (z.B. in externen Programmen wie MatLab) um andere mögliche äußere Einflüsse mit in Betracht zu ziehen. Möchte man mit der mySQL Datenbank Graphen darstellen wie mit dem RRDtool, so müsste man den Umweg gehen und die Daten als CSV-File exportieren und an Hand dieser dann mit einer anderen Software die Grafik erstellen (z.B. mit Excel).\\\\
Als Erkenntnisse zu den durchgeführten Versuchen ergeben sich, dass es kein allzu großes Problem darstellt mit den ausgewählten Komponenten funktionierende Schaltungen und die dazugehörige Software zu realisieren. Auch waren die erhaltenen Messergebnisse zufriedenstellend und ermöglichten einen sehr guten Überblick über die Gegebenheiten in den Messphasen (Temperaturen, Luftfeuchtigkeit etc.). Daher steht im weiteren Verlauf einem ersten Testeinsatz der einzelnen Schaltungen in dem Produktionsumfeld nichts im Wege. Weiterhin wäre noch angedacht, die einzelnen Schaltungen soweit zu erweitern, dass auch die Möglichkeiten zur Datenauswertung über verschiedenen Wege (z.B. über ein Netzwerk, Internet) möglich wäre.\\\\
Die Zukunftsaussichten der in dieser Arbeit getesteten Komponenten, sind durchwegs positiv zu sehen, da es immer mehr große Betrieb gibt, die den Einsatz von SBCs in ihrem Unternehmen testen. Weiterhin kann festgehalten werden, dass SBCs mit den dazugehörigen Sensoren, eine sehr gute Möglichkeit zur Datenerfassung in verschiedenen Bereichen bieten und dies auch noch zu einem Bruchteil der Kosten von professionellen Komponenten. Aus diesem Grund wird sich der Bereich der SBCs in der Zukunft sehr stark weiterentwickeln.
